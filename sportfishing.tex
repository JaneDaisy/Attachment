% Options for packages loaded elsewhere
\PassOptionsToPackage{unicode}{hyperref}
\PassOptionsToPackage{hyphens}{url}
%
\documentclass[
]{article}
\usepackage{amsmath,amssymb}
\usepackage{iftex}
\ifPDFTeX
  \usepackage[T1]{fontenc}
  \usepackage[utf8]{inputenc}
  \usepackage{textcomp} % provide euro and other symbols
\else % if luatex or xetex
  \usepackage{unicode-math} % this also loads fontspec
  \defaultfontfeatures{Scale=MatchLowercase}
  \defaultfontfeatures[\rmfamily]{Ligatures=TeX,Scale=1}
\fi
\usepackage{lmodern}
\ifPDFTeX\else
  % xetex/luatex font selection
\fi
% Use upquote if available, for straight quotes in verbatim environments
\IfFileExists{upquote.sty}{\usepackage{upquote}}{}
\IfFileExists{microtype.sty}{% use microtype if available
  \usepackage[]{microtype}
  \UseMicrotypeSet[protrusion]{basicmath} % disable protrusion for tt fonts
}{}
\makeatletter
\@ifundefined{KOMAClassName}{% if non-KOMA class
  \IfFileExists{parskip.sty}{%
    \usepackage{parskip}
  }{% else
    \setlength{\parindent}{0pt}
    \setlength{\parskip}{6pt plus 2pt minus 1pt}}
}{% if KOMA class
  \KOMAoptions{parskip=half}}
\makeatother
\usepackage{xcolor}
\usepackage[margin=1in]{geometry}
\usepackage{graphicx}
\makeatletter
\def\maxwidth{\ifdim\Gin@nat@width>\linewidth\linewidth\else\Gin@nat@width\fi}
\def\maxheight{\ifdim\Gin@nat@height>\textheight\textheight\else\Gin@nat@height\fi}
\makeatother
% Scale images if necessary, so that they will not overflow the page
% margins by default, and it is still possible to overwrite the defaults
% using explicit options in \includegraphics[width, height, ...]{}
\setkeys{Gin}{width=\maxwidth,height=\maxheight,keepaspectratio}
% Set default figure placement to htbp
\makeatletter
\def\fps@figure{htbp}
\makeatother
\setlength{\emergencystretch}{3em} % prevent overfull lines
\providecommand{\tightlist}{%
  \setlength{\itemsep}{0pt}\setlength{\parskip}{0pt}}
\setcounter{secnumdepth}{-\maxdimen} % remove section numbering
\ifLuaTeX
  \usepackage{selnolig}  % disable illegal ligatures
\fi
\usepackage{bookmark}
\IfFileExists{xurl.sty}{\usepackage{xurl}}{} % add URL line breaks if available
\urlstyle{same}
\hypersetup{
  pdftitle={Sport Fishing Kenya Marine Fisheries Institute},
  pdfauthor={Nimrod Ishmael},
  hidelinks,
  pdfcreator={LaTeX via pandoc}}

\title{Sport Fishing Kenya Marine Fisheries Institute}
\author{Nimrod Ishmael}
\date{Completion Date February 20, 2025}

\begin{document}
\maketitle

\subsection{Introduction}\label{introduction}

The Project is aimed answering the following research questions and
objectives

\textbf{Objectives}

\begin{itemize}
\tightlist
\item
  To determine annual landings, catch trends per year/ location/fish
  type from KASA catch records from 1991 to 2015.
\item
  To determine the nominal Catch per Unit Effort of the sailfish along
  the Kenyan coast from the sport fishing data.
\item
  To determine the temporal variability of sailfish catches across
  various years.
\end{itemize}

\textbf{Research Questions}

\begin{itemize}
\tightlist
\item
  What are the annual landings for the billfish species over the period
  obtained from the source data?
\item
  What are the catch trends per year/ location/fish type?
\item
  What are the trends in the catch per unit effort for the billfish
  based on the sport fisheries data?
\end{itemize}

\subsection{Data Wranging}\label{data-wranging}

This section will cover ;

\begin{itemize}
\tightlist
\item
  Downloading data

  \begin{itemize}
  \tightlist
  \item
    Use the Google Sheet API to download the source data
  \end{itemize}
\item
  Data Cleaning Procedure

  \begin{itemize}
  \tightlist
  \item
    Exclude none numerical entries on the weight variable
  \item
    Clean up misplaced fish type
  \item
    Exclude all entries with zero weight
  \item
    Tag wrong entries for correction \emph{i.e tagged number cannot be
    greater than number of fish recorded}
  \end{itemize}
\item
  Manipulation of data

  \begin{itemize}
  \tightlist
  \item
    Calculate the average weight of fish
  \item
    Calculate the release weight
  \end{itemize}
\end{itemize}

\newpage

\subsection{Explanatory Data Analysis}\label{explanatory-data-analysis}

This section cover descriptive analysis for the sport fishing source
data in different fishing zones along the coast region from
\textbf{1991} to \textbf{2015}.

\subsubsection{Annual Fish Weight
Trends}\label{annual-fish-weight-trends}

The graph below indicates the annual recorded fish weight in coast
region since 1991 to 2015 .

\begin{center}\includegraphics{sportfishing_files/figure-latex/unnamed-chunk-2-1} \end{center}

\textbf{Summary Statistics}

The was an average weight of \textbf{188 kgs} with the highest and
minimum weight of \textbf{301 kgs}, \textbf{18 kgs} respectively.

\textbf{The General Trend}

The graph above displays the total fish caught over the the years. From
the graph it is evident that maximum fish weight caught was realized in
both year 1992 and 2009 as well as the minimum weight caught was in year
2015. The graph illustrates continuous decrease in total fish caught
from 2008 to 2015 and a relative fluctuation in weight caught between
years 1993 and 2009.

To understand the trend in the fish weight over the years we calculate
the lag weight between the years to get the percentage change from one
year to the next. The results were displayed in both an histogram and a
fitted generalized model using \texttt{loess\ function}.

\newpage

\textbf{Lag Weight Bar Graph }

The bar graph indicates the change in caught fish weight from one year
to the next.

\includegraphics{sportfishing_files/figure-latex/unnamed-chunk-3-1.pdf}

From the bar graph above it is evident that there was a spike increase
in weight caught between the years, \texttt{1991\ and\ 1992},
\texttt{2006\ and\ 2007} and \texttt{2008\ and\ 2009} with increased
percent of \texttt{71.2\%} , \texttt{78.1\%} and \texttt{77.2\%}
respectively. There was a stiff decrease in the weight caught between
years \texttt{2005\ and\ 2006} and \texttt{2014\ and\ 2015} with a total
percentage of \texttt{-62.6\ \%} and \texttt{-78.2\%} respectrively.

\newpage

\textbf{Caught Fish Weight Trend}

A generalized model was fitted to the data to map out the trend in fish
weight over the years.

\begin{center}\includegraphics{sportfishing_files/figure-latex/unnamed-chunk-4-1} \end{center}

The graph above indicates a decrease in weight caught between year 1991
to 1995 and a steady increase from 1995 to 2007 from which there was a
continuous sharp decrease in weight.

\newpage

\subsubsection{Fish Weight Zone Trends}\label{fish-weight-zone-trends}

This section focus on the trends of caught fish weight per fishing zone.

\textbf{Total Fish Weight Caught Per Zone}

The bar graph illustrates total weight caught per zone between 1991 and
2015.

\begin{center}\includegraphics{sportfishing_files/figure-latex/unnamed-chunk-5-1} \end{center}

Shimoni, Mombasa and Lamu fishing zone's show a gradual decrease in
weight of fish caught over the years with Kilifi zone having a
fluctuation and a gap of unrecorded weight for two years. Where as
Watamu zone had a steep decrease in fishing weight from year 2010 to
2015.

To get the trend of fishing weight between the recorded years for all
zones we calculate the lag weight and fit a generalized model on the
data

\newpage

\textbf{Fish Weight Trend per Zone}

The graph below displays fitted generalized model to map the trends of
fishing weight for different zones over the years.

\begin{center}\includegraphics{sportfishing_files/figure-latex/unnamed-chunk-6-1} \end{center}

The following are the observation from the trends.

\begin{itemize}
\tightlist
\item
  Kilifi fishing ground had a decrease in fishing weight between 1991 to
  2004 and steady increase henceforth to year 2011.
\item
  Lamu and Kikwayu fishing zones registered a continuous decrease in
  fishing weight over the years.
\item
  Malindi and Ngomeni zones had a decrease in fishing weight between the
  years 1991 to 1996. The zone also registered a steady increase in fish
  weight in preceding year's up to year 2006 and a sharp decrease
  henceforth.
\item
  Mombasa and Mtwapa zones had a steady fishing weight with an increase
  in caught fish weight in year 2001 to 2007 and a sharp decline
  henceforth.
\item
  Shimoni zone had a continuous decrease in fishing weight over the
  years.
\item
  Watamu fishing zone registered a decrease in caught fish weight
  between 1991 and 1996. From 1997 the zone posted an increase in weight
  to year 2007 and a decrease hence forth.
\end{itemize}

\newpage

\subsubsection{Fish Type Weight per
Zone}\label{fish-type-weight-per-zone}

The graph shows total weight caught per fish type in different fishing
zones

\includegraphics{sportfishing_files/figure-latex/unnamed-chunk-7-1.pdf}

The graph above shows different fish types in several fishing grounds.
From the graph its evident that;

\begin{itemize}
\tightlist
\item
  Yellowfin Tuna and sailfish were the predominant caught fish in kilifi
  zone.
\item
  Yellowfin Tuna, Wahoo and sailfish were the most caught fish in Lamu,
  kikwayu zone.
\item
  In Ngomeni, Malindi zone sailfish was the common species caught in the
  region.
\item
  Sailfish, Yellow Tuna and Dorado were the predominat fish types caught
  in Mtwapa, Mombasa zone.
\item
  Striped Marlin and sailfish were the common fish caught in Shimoni
  fishing zone and
\item
  In watamu zone sailfish was the most common caught fish in the area.
\end{itemize}

\newpage

\textbf{Fish Type Trends}

To better understand the trends of weight per fish type we fit a
generalized model to map the trends effectively this is displayed in the
graph attached below.

\begin{center}\includegraphics{sportfishing_files/figure-latex/unnamed-chunk-8-1} \end{center}

\newpage

\subsubsection{Sail Fish Descriptives}\label{sail-fish-descriptives}

This section will cover variability in caught \textbf{Sailfish} weight
between the year 1991 to 2015.

\begin{center}\includegraphics{sportfishing_files/figure-latex/unnamed-chunk-9-1} \end{center}

\textbf{Sail Fish Summary Statistics}

The average recorded weight for sailfish was \textbf{58 kgs} with the
highest and minimum weight of \textbf{186 kgs} and \textbf{4 kgs}
respectively.

\newpage

\paragraph{Sail Fish Trends}\label{sail-fish-trends}

To better understand the trends in weight of the sailfish we fit a
generalized model using the \texttt{loess\ function} to map the trends.

\textbf{General Sailfish Weight Trend}

This section concentrates on mapping out the overall trend of sailfish
weight over the years.

\emph{Please Note: Year 1991 has been excluded in the plot because it an
outlier which skews the analysis}

\begin{center}\includegraphics{sportfishing_files/figure-latex/unnamed-chunk-10-1} \end{center}

The graph above shows trends in weight of sailfish from 1992 to 2015.
The graph indicates an upward trend in caught weight of fish from 1992
to 2007 from which their is an alarming steep decrease in the weight.

\newpage

\textbf{Sailfish Weight Trend per Zone}

To better understand the variability of sail fish we map its trend in
different fishing regions, which are illustrated by the graphs below.

\begin{center}\includegraphics{sportfishing_files/figure-latex/unnamed-chunk-11-1} \end{center}

The graph above shows different trends in the sailfish weight recorded
over the years;

\begin{itemize}
\tightlist
\item
  \textbf{Kilifi Region}

  \begin{itemize}
  \tightlist
  \item
    A sharp decrease in sailfish weight was recorded between year 1991
    to 1996 from which the weight tremendously increses up to 2005,
    henceforth a steep decline in weight recorded in preceding years.
  \end{itemize}
\item
  \textbf{Lamu Kikwayu Region}

  \begin{itemize}
  \tightlist
  \item
    A tremendous increade in sailfish weight was recorded between 1991
    to around 1999 from which the weigh steeply decreased in preceding
    years
  \end{itemize}
\item
  \textbf{Malindi Ngomeni Region}

  \begin{itemize}
  \tightlist
  \item
    A sharp decrease in recorded weight sailfish was recorded between
    the year 1991 to around 1996 from which the weight increased steadly
    to yeat 2007 and a decrease hence forth.
  \end{itemize}
\item
  \textbf{Mombasa Mtwapa Region}

  \begin{itemize}
  \tightlist
  \item
    A steady decrease in sailfish recorded weight was realized between
    year 1991 to 2001 from which a steep increase in weight was recorded
    up to year 2007 hence a sharp decline over the next couple of years.
  \end{itemize}
\end{itemize}

\newpage

\begin{itemize}
\tightlist
\item
  \textbf{Shimoni Region}

  \begin{itemize}
  \tightlist
  \item
    A fluctuating weight of sailfish was recorded between 1991 to 2002
    from which the weight tremendously decreased
  \end{itemize}
\item
  \textbf{Watamu Region}

  \begin{itemize}
  \tightlist
  \item
    An increase in sailfish weight was recorded from 1991 to 2003 in
    Watamu region from which the weigh steeply decreased over the next
    preceding years
  \end{itemize}
\end{itemize}

\newpage

\subsection{Inferentials Analysis}\label{inferentials-analysis}

\subsubsection{Time Series}\label{time-series}

\textbf{Visualize Time-series}

visualization helps discover underlying time series behavior/ trends as
well as detecting anomalies. To map the time series trends we plot the
data by use of ggplot2 function, \texttt{autoplot()}

\begin{center}\includegraphics{sportfishing_files/figure-latex/unnamed-chunk-13-1} \end{center}

The graph above does not illustrate any trend in fish weight in other
words it is cyclic. To zero in and determine if they could be any trend
we calculate lag.

\textbf{Lag Calculation}

lag is the difference of current years fish weight value and the former
year. The attached equation better explains lag
\textbf{\(lag_{t} = y_{t+1} - y_{t}\)}

\begin{center}\includegraphics{sportfishing_files/figure-latex/unnamed-chunk-15-1} \end{center}

The graph above shows cyclic trends in that it does not have a regular
trend of caught fish weight in reference to years

\newpage

\textbf{Decompose Time Series}

We decompose the time series object into trend, seasonal and residual
component to further get insights on the trend of caught fish weight.

\begin{center}\includegraphics{sportfishing_files/figure-latex/unnamed-chunk-16-1} \end{center}

From the decomposed time series object on fish weight it is clear that
the trend on fish weigh does not follow any defined pattern hence cyclic
in its nature. Therefore, we conclude that fish weight over the years do
not have any defined trend as it increases and reduces irregulary.

\newpage

\textbf{Forecast}

Forecasting is basically predicting the period past the current time. In
this section we will use naive method to predict the trend of fish
weight for the next 20 years.

\textbf{Fish Weight Forecast}

\begin{center}\includegraphics{sportfishing_files/figure-latex/unnamed-chunk-17-1} \end{center}

The graph above indicates forecast fish weight highlighted in blue. From
the graph it is evident that the caught fish weight over the next 20
years will increase and reduce irregularly following a cyclic time
series trend.

\subsubsection{Regression}\label{regression}

Regression is mostly used to explore relationship of two or more
variables. This relationship can be mapped in either linear or
non-linear relationship. The most common is the linear/Multiple
regression which follows the equation
\(Y = B{_0} + B{_1}X{_1} + B{_2}X{_2}+ . . . + B{_k}X{_k}\)

\newpage

\textbf{Distribution of Weight}

Every parametric distribution follows certain assumptions.In our case
regression model follows normality assumption and this can be identified
though mapping density plot for the fish weight.

\begin{center}\includegraphics{sportfishing_files/figure-latex/unnamed-chunk-19-1} \end{center}

The density plot above indicates that distribution of fish weight is
skewed to the left. Which breaks the normality assumption for multiple
linear regression assumptions.

\newpage

\textbf{Transformed Fish Weight}

In quest to achieve normal distribution of the response variable
\emph{(i.e fish weight)} we calculate the logarithm of fish weight to
base 10 \emph{(\(log{_10}\))} and plot a density graph to check the
distribution of the transformed weight.

\begin{center}\includegraphics{sportfishing_files/figure-latex/unnamed-chunk-20-1} \end{center}

The above density plot indicates tow density plots. The blue line shows
normal distribution whereas the filled grey density plot shows the
distribution of transformed fish weight.

\newpage

\textbf{Normality Test}

From the graph it is unclear if the data follows normal distribution,
therefore we carry a statistical normality test which can be done by
either;

\begin{itemize}
\tightlist
\item
  Shapiro test \emph{or}
\item
  One-sample Kolmogorov-Smirnov test
\end{itemize}

\emph{One-sample Kolmogorov-Smirnov test}

\begin{verbatim}
## 
##  Asymptotic one-sample Kolmogorov-Smirnov test
## 
## data:  inferential_data$total_weight
## D = 0.99952, p-value < 2.2e-16
## alternative hypothesis: two-sided
\end{verbatim}

\emph{Shapiro-Wilk Normality Test}

\begin{verbatim}
## 
##  Shapiro-Wilk normality test
## 
## data:  inferential_data$log_weight
## W = 0.99601, p-value = 0.0002222
\end{verbatim}

P value for normality test in \textbf{Shapiro Test} and
\textbf{Kolmogorov-Smirnov Test} is \textbf{0.0002222} and
\textbf{2.2e-16} respectively, which is less than 0.05 significance
level. Therefore, we reject the null hypothesis that the data assumes
normal distribution.

The two test confirm that we cannot run a multiple linear regression on
sport fishing data hence switching to non-parametric methods.

\textbf{Kruskal-Wallis Test}

Kruskal-Wallis Test is the alternative test for Analysis of Variance
(ANOVA) in parametric distribution. \emph{Please note the test is a
distribution free in other words is a non-parametric method}

\begin{verbatim}
## 
##  Kruskal-Wallis rank sum test
## 
## data:  .
## Kruskal-Wallis chi-squared = 8124.6, df = 5, p-value < 2.2e-16
\end{verbatim}

From the output we can see that the chi-squared test statistic is 8124.6
and the corresponding p-value is 2.2e-16. Since this p-value is less
than the .05 significance level. We conclude that indeed there is a
statistical significant difference between the reported fish weight in
the five fishing zones.

\end{document}
